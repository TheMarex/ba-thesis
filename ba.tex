% Use UTF-8 encoding!!
% (c) Stefan Ulbrich, 2012

\documentclass[english,ngerman]{KITreprt}


%% -------------------------------
%% |  Information for PDF file   |
%% -------------------------------
\hypersetup{
 pdfauthor={Patrick Niklaus},
 pdftitle={Bachelor Thesis: Walking pattern generation for humanoid bipedal robots},
 pdfsubject={Not set},
 pdfkeywords={Not set}
}


%% --------------------------------
%% Obligatory Parameters:
%% --------------------------------

\renewcommand{\myname}{Patrick Niklaus}
\renewcommand{\mythesis}{\bachelorsthesis} %\mastersthesis, \bachelorsthesis, \protocol, \studienarbeit, \diplomarbeit
\renewcommand{\mytitle}{ Dynamically Stable Walking For Humanoid Bipedal Robots Based On Walking Patterns }
%\renewcommand{\myshorttitle}{Die offizielle \LaTeX-Vorlage des HIS}
\renewcommand{\myshorttitle}{}
\cfoot{\mytitle}

\renewcommand{\timestart}{1. April 2014}
\renewcommand{\timeend}{1. Oktober 2014}
%\renewcommand{\timeend}{\iflanguage{english}{February 7\textsuperscript{th},}{7. Februar} 2010}
\newcommand{\advisor}{Dr. Júlia Borràs Sol}
%\newcommand{\advisortwo}{Nur wenn n\"otig}

\newcommand{\clr}[2]{{\color{#1}{#2}}}
\newcommand{\todo}[1]{\marginpar{\clr{red}{#1}}}
\newcommand{\slpy}[1]{{\sloppy #1}}
\newcommand{\fixme}[1]{{\color{red}{FIXME: #1}}}

\graphicspath{{./images/}}


\begin{document}


\selectlanguage{english}

%\selectlanguage{ngerman}


\maketitle

\tableofcontents

\chapter{Introduction}\label{introduction}

motivation, and a bit of overview of humanoid walking. I recommend to
leave it for later, start with the sections that you feel its easier to
write (usually, the ones that have more content).

\begin{itemize}
\itemsep1pt\parskip0pt\parsep0pt
\item
  motivation:

  \begin{itemize}
  \itemsep1pt\parskip0pt\parsep0pt
  \item
    navigating in human environments
  \end{itemize}
\item
  walking in humans:

  \begin{itemize}
  \itemsep1pt\parskip0pt\parsep0pt
  \item
    CoM movement, gait phases, differences to what we do here
  \end{itemize}
\item
  static vs.~dynamic walking
\item
  overview of models used for dynamic walking
\end{itemize}

\chapter{Models for humanoid walking}\label{models-for-humanoid-walking}

\section{The Linear Inverted Pendulum
Model}\label{the-linear-inverted-pendulum-model}

\todo{Use different name for CoM, $p$ will be rather used for the ZMP, maybe $c$?}

\todo{picture of 3D-LIPM}

A simple model for describing the dynamics of a bipedal robot during
single support phase is the 3D inverted pendulum. We reduce the body of
the robot to a point-mass at the center of mass and replace the support
leg by a mass-less telescopic leg which is fixed at a point on the
supporting foot. Initially this will yield non-linear equations that
will be hard to control. Howevery by constraining the movement of the
inverted pendulum to a fixed plane, we can derive a linear dynamic
system. This model called the 3D \emph{linear} inverted pendulum model
(short \emph{3D-LIPM}).

\subsection{The inverted pendulum}\label{the-inverted-pendulum}

To describe the dynamics of the inverted pendulum we are mainly
interested in the effect a given actuator torque has on the movement of
the pendulum.

For simplicity we assume that the base of the pendulum is fixed at the
origin of the current cartesian coordinate system. Thus we can describe
the position inverted pendulum by a vector $p = (x, y, z)$. We are going
to introduce an appropriate (generalized) coordinate system
$q = (\theta_R, \theta_P, r)$ to get an easy description of our actuator
torques: Let $m$ be the mass of the pendulum and $r$ the length of the
telescopic leg. $\theta_P$ and $\theta_R$ describe the corresponding
roll and pitch angles of the pose of the pendulum.
\todo{add image with angles here}

Now we need to find a mapping between forces in the cartesian coordinate
system and the generalized forces (the actuator torques). Let
$\Phi: \mathbb{R}^3 \longrightarrow \mathbb{R}^3, (\theta_R, \theta_P, r) \mapsto (x, y, z)$
be a function that maps the generalized coordinates to the cartesian
coordinates. Then the jacobian $J_\Phi = \frac{\partial p}{\partial q}$
maps the \emph{generalized velocites} to \emph{cartesian velocites}.
Furthermore we know that the transpose $J_\Phi^T$ maps \emph{cartesian
forces} $F = m (\ddot x, \ddot y, \ddot z)$ to \emph{generalized forces}
$(\tau_r, \tau_p, f)$.

We write $x$, $y$ and $z$ in terms of our generalized coordinates to
compute the corresponding jacobian $J_\Phi$. From the fact that the
$\theta_P$ is the angle between the projection of $p$ onto the
$xz$-plane and $p$ and $\theta_R$ the angle between $p$ and the
projection onto the $yz$ plane we can derive the following equations
\todo{reference paper}:

\begin{equation}
\begin{array}{lcll} \label{eq:lip-xyz}
x & = & r \cdot \sin \theta_P & =: r \cdot s_P\\
y & = & -r \cdot \sin \theta_R & =: -r \cdot s_R \\
z & = & \sqrt{r^2 - x^2 - y^2} = r \cdot \sqrt{1 - s_P^2 - s_R^2} & \\
\end{array}
\end{equation}

From which we can compute the jacobian by partial derivation:

\begin{equation} \label{eq:lip-jacobian}
J = \frac{\partial p}{\partial q} = \left( \begin{array}{rcl}
0 & r \cdot c_P & s_P \\
-r \cdot c_R & 0 & s_P \\
\frac{2 \cdot r \cdot s_P c_P}{\sqrt{1 - s_P^2 - s_R^2}} & \frac{2 \cdot r \cdot s_R c_R}{\sqrt{1 - s_P^2 - s_R^2}} & \sqrt{1 - s_P^2 - s_R^2}\\
\end{array}
\right)
\end{equation}

Using the equation of motion as given by

\begin{equation}
\begin{array}{lcr}
F & = & (J^T)^{-1} \Gamma + f_g \\
m \cdot
\left(\begin{array}{c}
\ddot x \\
\ddot y \\
\ddot z \\
\end{array}\right)
& = & (J^T)^{-1}
\left(\begin{array}{c}
\tau_R \\
\tau_P \\
f \\
\end{array}\right)
+
\left(\begin{array}{c}
0 \\
0 \\
-m \cdot g \\
\end{array}\right) \\
\end{array}
\end{equation}

and equations \ref{eq:lip-jacobian} and \ref{eq:lip-xyz} we can derive
the following equations:

\begin{equation} \label{eq:lip-dyn-y}
m(-z\ddot{y} + y\ddot{z}) = \frac{\sqrt{1 - s_P^2 - s_R^2}}{c_R} \cdot \tau_R + m g y
\end{equation}

\begin{equation} \label{eq:lip-dyn-x}
m(z\ddot{x} - x\ddot{z}) = \frac{\sqrt{1 - s_P^2 - s_R^2}}{c_P} \cdot \tau_P + m g x
\end{equation}

Observe that the terms of the left-hand side are not linear. To remove
that non-linearity we are going to use the \emph{linear} inverted
pendulum model.

\subsection{Linearization}\label{linearization}

In a man-made environment it is fair to assume that the ground a robot
will walk on can be approximate by a slightly sloped plane. In most
cases it can even assumed that there is no slope at all.

The basic assumption in the next section will be that the CoM will have
a \emph{constant displacement} with regard to our ground plane. Thus we
can constrain the movement of the CoM to a plane that is parallel to the
ground plane. Note that this assumption is, depending on the walking
speed, only approximately true for human walking as shown by Orendurff
et. al. For slow to fast walking ($0.7$ m/s and $1.6$ m/s respectively)
the average displacement in $z$-direction was found to be between
$2.7cm$ and $4.81$ cm. While the walking patterns generated based on the
LIP-model will guarantee dynamic stability, they might not look natural
with regard to human walking.

\todo{cite Orendurff}

We are going to constrain the $z$ coordinate of our inverted pendulum to
a plane with normal vector $(k_x, k_y, -1)$ and $z$-displacement $z_c$:

\begin{equation} \label{eq:lip-z-plane}
z = k_x \cdot x + k_y \cdot y + z_c
\end{equation}

Subsequently the second derivative of $z$ can be described by:

\begin{equation} \label{eq:lip-z-div}
\ddot{z} = k_x \cdot \ddot{x} + k_y \cdot \ddot{y}
\end{equation}

Substituing \ref{eq:lip-z-plane} and \ref{eq:lip-z-div} into the
equations \ref{eq:lip-dyn-y} and \ref{eq:lip-dyn-x} yields the following
equations:

\begin{equation}
\ddot{y} = \frac{g}{z_c} y - \frac{k_x}{z_c} (x\ddot{y} - \ddot{x}y) - m z_c \cdot \tau_R \cdot \frac{\sqrt{1 - s_P^2 - s_R^2}}{c_R}
\end{equation}

\begin{equation}
\ddot{x} = \frac{g}{z_c} x + \frac{k_y}{z_c} (x\ddot{y} - \ddot{x}y) + m z_c \cdot \tau_P \cdot \frac{\sqrt{1 - s_P^2 - s_R^2}}{c_P}
\end{equation}

The term $x\ddot{y} - \ddot{x}y$ that is part of both equations is still
causing the equations to be non-linear. To make this equations linear we
will assume that our ground plane has no slope, thus $k_x = k_y = 0$ and
the non-linear terms will vanish.

Another problem is that the actuator torques $\tau_R$ and $\tau_P$ both
have non-linear factors $\frac{\sqrt{1 - s_P^2 - s_R^2}}{c_R}$ and
$\frac{\sqrt{1 - s_P^2 - s_R^2}}{c_P}$ respectively. This can be solved
by substituting with the following \emph{virtual inputs}:

\begin{equation}
\tau_P \cdot \frac{\sqrt{1 - s_P^2 - s_R^2}}{c_P} = u_P
\end{equation}

\begin{equation}
\tau_R \cdot \frac{\sqrt{1 - s_P^2 - s_R^2}}{c_R} = u_R
\end{equation}

Which yields our final description of the dynamics:

\begin{equation} \label{eq:lip-y}
\ddot{y} = \frac{g}{z_c} y - \frac{u_R}{m z_c}
\end{equation}

\begin{equation} \label{eq:lip-x}
\ddot{x} = \frac{g}{z_c} x + \frac{u_R}{m z_c}
\end{equation}

\todo{include pattern generation just based on 3D-LIPM, I don't understand how they derived the controller}

\section{The Zero Moment Point}\label{the-zero-moment-point}

A very popular approach to humanoid walking are schemes based on the
Zero Moment Point. One reason for that might be that it is very simple
to describe constrains for dynamic stability using this reference point.
As long as the following condition is met we will have full ground
contact of our support foot and thus can realize dynamically stable
walking: \emph{The ZMP is strictly inside the support polygone of the
support foot.}

For flat ground contact of our support foot with the floor the ZMP
corresponds with the position of the center of pressure (CoP). Indeed,
some author (notably Pratt) prefer to use the term CoP instead of ZMP.

The CoP of an object in contact with the ground can be computed as the
sum of all contact points $p_1, \dots, p_n$ weighted by the forces in
$z$-direction $f_{1z}, \dots, f_{nz}$ that is applied:

\begin{equation} \label{eq:zmp-definition}
p := \frac{\sum^N_{i=1}p_i f_{iz}}{\sum^N_{i=1} f_{iz}}
\end{equation}

An important fact (and the origin of its name) is that there are no
torques around the $x$ and $y$ axis at the ZMP:

\begin{equation}
\tau = \sum^N_{i=1} (p_i - p) \times f_i
\end{equation}

Splitting that up into each component using the definition of the cross
product yields:

\begin{equation}
\tau_x = \sum^N_{i=1} (p_{iy} - p_y) f_{iz} - \overbrace{(p_{iz} - p_z)}^{=0} f_{iy}
\end{equation}

\begin{equation}
\tau_y = \sum^N_{i=1} \overbrace{(p_{iz} - p_z)}^{=0} f_{ix} - (p_{ix} - p_x) f_{iz}
\end{equation}

\begin{equation}
\tau_z = \sum^N_{i=1} (p_{ix} - p_x) f_{iy} - (p_{iy} - p_y) f_{ix}
\end{equation}

Since we have flat ground contact, all contact points have the same
$z$-coordinate as the ZMP, thus we can simplify $\tau_x$ and $\tau_y$
to:

\begin{equation} \label{eq:zmp-torque-x}
\tau_x = \sum^N_{i=1} (p_{iy} - p_y) f_{iz} = \sum^N_{i=1} (p_{iy} f_{iz}) - (\sum^N_{i=0} f_{iz}) \cdot p_y
\end{equation}

\begin{equation}\label{eq:zmp-torque-y}
\tau_y = \sum^N_{i=1} - (p_{ix} - p_x) f_{iz} = \sum^N_{i=1} - (p_{ix} f_{iz}) + (\sum^N_{i=0} f_{iz}) \cdot p_x
\end{equation}

Furthermore we can use the corresponding components $p_x$ and $p_y$ from
the definition of the ZMP \ref{eq:zmp-definition} and substitude in the
equations \ref{eq:zmp-torque-x} and \ref{eq:zmp-torque-y}.

This will yield: $\tau_x = \tau_y = 0$.

Please note that $\tau_z$ will in general not be zero, nonetheless in
case of straight walking it is often assumed to be zero as well.

\section{The table-cart model}\label{section:table-cart}

The table-cart model is equivalent to the 3D-LIPM model discussed
before, but somewhat more intuitive for computing the resulting ZMP from
an CoM motion. The model consists of an (infinitely) large mass-less
table of height $z_c$, while the foot of the table has the shape of the
support polygone. Given a frictionless cart with mass $m$ that moves on
the table we can compute the resulting ZMP in the support foot. Please
note that the 3D-dimensional model is equivalent to having two
independent tables with two carts each in the $xz$ and $yz$-plane
respectively. First of all, lets compute the torque $\tau_x$ and
$\tau_y$ around the x-axis and y-axis at the ZMP on the support foot.

\begin{equation}
\tau_y = \overbrace{-m g (c_x - p_x)}^{\text{torque due to gravity}} + \overbrace{m \ddot{x} \cdot z_c}^{\text{torque due to acceleration of cart}}
\end{equation}

\begin{equation}
\tau_x = -m g (c_y - p_y) + m \ddot{y} \cdot z_c
\end{equation}

Please note the similarity to the equations \ref{eq:lip-y} and
\ref{eq:lip-x} when assuming the base of the pendulum is located at $p$.
If we now use the property of the ZMP that the torque around the $x$ and
$y$-axis is zero, we can solve for the ZMP position $p$:

\begin{equation} \label{eq:zmp-x}
p_x = c_x - \frac{z_c}{g} \ddot{c_x}
\end{equation}

\begin{equation} \label{eq:zmp-y}
p_y = c_y - \frac{z_c}{g} \ddot{c_y}
\end{equation}

\todo{Maybe describe the full-body methode to compute the ZMP}

\section{Simulating rigid body
dynamics}\label{simulating-rigid-body-dynamics}

\todo{short introduction how that works and some problems with the approach}

\chapter{Pattern generator}\label{pattern-generator}

To generate a walking pattern for a bipedal robot two basic approaches
are common:

\begin{enumerate}
\def\labelenumi{\arabic{enumi}.}
\item
  Generate (or modify) foot trajectories that realize a prescribed
  trajectory of the CoM
\item
  Generate a CoM trajectory for prescribed foot trajectories
\end{enumerate}

The first approach is generally used for implementing pattern generators
soley based on the 3D-LIPM model. \todo{citation needed}

The second approach is the more versatile one, since it is easy to
incorporate constrains of our environment (e.g.~only limited foot holds)
in the input of the pattern generator. However care must be taken while
chosing adequate step width and step length parameters for the foot
trajectory, so that they can actually be realized by the robot.

The pattern generator proposed by Kajita et al. \todo{add citation}
based on Preview Control realizes the second approach. We will discuss
the theoretical background of this pattern generator here in more
detail, since all pattern that we used where generated that way.

\section{Computing the CoM from a reference
ZMP}\label{computing-the-com-from-a-reference-zmp}

As we saw in the section \ref{section:table-cart} it is easy to compute
the resulting ZMP given the CoM and its acceleration. However for
generating the walking pattern, we want to compute the CoM trajectory
from a given ZMP. If you rearange the equations \ref{eq:zmp-x} and
\ref{eq:zmp-y} you see that we have to solve a second order differential
equations:

\begin{equation} \label{eq:com-x}
c_x = \frac{z_c}{g} \cdot \ddot{c_x} + p_x
\end{equation}

\begin{equation} \label{eq:com-y}
c_y = \frac{z_c}{g} \cdot \ddot{c_y} + p_y
\end{equation}

There are several ways to solve this differential equations, for example
by transforming them to the frequency-domain. This however would mean,
the ZMP trajectory needs to be transformed to the frequency domain as
well, e.g.~using Fast Fourier Transformation. This has two main
problems:

\begin{enumerate}
\def\labelenumi{\arabic{enumi}.}
\item
  It has a significant computational overhead. (For FFT the additional
  runtime would be in $O(n \log n)$)
\item
  We need to know the whole ZMP trajectory in advance.
\end{enumerate}

Instead Kajita et. al. chose to define a dynamic system in the time
domain that describes the CoM movement.

\subsection{Pattern generation as dynamic
system}\label{pattern-generation-as-dynamic-system}

\todo{maybe do a formal introduction into dynamic system and the state space approach}

For simplicity we will only focus on the dynamic description of one
dimension, as the other one is analogous. To transform the equations to
a strictly proper dynamical system, we need to determine the state
vector of our system. For the table-cart model it suffices to know the
position, velocity and acceleration of the cart. Thus the state-vector
is defined as $x = (c_x, \dot{c_x}, \ddot{c_x})$. We can define the
evolution of the state vector as follows:

\begin{equation} \label{eq:dyn-system}
\frac{d}{dt} \left(\begin{array}{c}
c_x \\
\dot{c_x} \\
\ddot{c_x} \\
\end{array} \right)
=
\overbrace{
\left(\begin{array}{ccc}
0 & 1 & 0\\
0 & 0 & 1 \\
0 & 0 & 0 \\
\end{array}\right)
}^{ =: A_0}
\cdot
\left(\begin{array}{c}
c_x \\
\dot{c_x} \\
\ddot{c_x} \\
\end{array}\right)
+
\overbrace{
\left(\begin{array}{ccc}
0 & 0 & 0\\
0 & 0 & 0\\
1 & 0 & 0\\
\end{array}\right)
}^{ =: B_0}
u
\end{equation}

As you can see the jerk of the CoM was introduced as an input
$u_x = \frac{d}{dt} \ddot{c_x}$ into the dynamic system.

We use equation \ref{eq:zmp-x} to calculate the actual output of the
dynamic system the resulting zmp, that will be controlled:

\begin{equation} \label{eq:zmp-x-output}
p_x =
\left(\begin{array}{ccc}
1 & 0 & \frac{-z_c}{g} \\
\end{array}\right)
\cdot
\left(\begin{array}{c}
c_x \\
\dot{c_x} \\
\ddot{c_x} \\
\end{array}\right)
\end{equation}

Using this formulation of the dynamic system we need to derive the
evolution of our state vector using the state-transition matrix. Since
our input ZMP trajectory will consist of discrete samples at equal time
intervals $T$ we define the discrete state as $x[k] := x(k \cdot T)$.
Please note that this system is a linear time-invariant system (LTI),
and both matrices $A_0$ and $B_0$ are constant. We can therefore use the
standart approach to solve this system using the equation:

\begin{equation}
x(t) = e^{A_0 \cdot (t - \tau)} x(\tau) + \int^t_\tau e^{A_0 \cdot (t - \lambda)} B_0 u(\lambda) d\lambda
\end{equation}

In our discrete case that becomes:

\begin{eqnarray} \label{eq:state-transition-discrete}
x[k+1] & = & e^{A_0 \cdot ((k+1)T - kT)} x[k] + \int^{(k+1)T}_{kT} e^{A_0 \cdot ((k+1)T - \lambda)} B_0 u(\lambda) d\lambda \\
       & = & e^{A_0 \cdot T} x[k] + \left(\int^{(k+1)T}_{kT} e^{A_0 \cdot ((k+1)T - \lambda)} d\lambda \right) \cdot B_0 u[k]\\
       & = & e^{A_0 \cdot T} x[k] + \left(\int^{0}_{T} e^{A_0 \cdot \lambda} d\lambda\right) \cdot B_0 u[k]
\end{eqnarray}

Keep in mind that $u(\lambda) = u[k], \lambda \in (kT, (k+1)T)$ so we
can move it outside of the integral. Let us first compute a general
solution for the matrix exponential $e^{A_0 \cdot t}$. It is easy to see
that $A_0$ is nilpotent and $A_0^3 = 0$, thus the computation simpilfies
to the following:

\begin{equation}
e^{A_0 t} := \sum^{\infty}_{i=0} \frac{(A_0 \cdot t)^i}{i!} = I + A_0 \cdot t + A_0^2 \cdot \frac{t^2}{2} + 0
=
\left(\begin{array}{ccc}
1 & t & \frac{t^2}{2}\\
0 & 1 & t \\
0 & 0 & 1 \\
\end{array}\right)
\end{equation}

Using this solution computing the integral in
\ref{eq:state-transition-discrete} is quite easy:

\begin{equation}
\int^{0}_{T} e^{A_0 \cdot \lambda} d\lambda =  -\int^{T}_{0} \left(\begin{array}{ccc}1 & t & \frac{t^2}{2}\\0 & 1 & t\\ 0 & 0 & 1\end{array}\right) dt
                                            =  -\left.\left(\begin{array}{ccc} %
                                                          t & \frac{t^2}{2} & \frac{t^3}{6} \\ %
                                                          0 & t             & \frac{t^2}{2} \\ %
                                                          0 & 0             & t %
                                                 \end{array}\right)\right|_{0}^{T}\\
                                            =   \left(\begin{array}{ccc} %
                                                          T & \frac{t^2}{2} & \frac{T^3}{6} \\ %
                                                          0 & T             & \frac{T^2}{2} \\ %
                                                          0 & 0             & T %
                                                \end{array}\right)
\end{equation}

Substituting the results in \ref{eq:state-transition-discrete} yields:

\begin{eqnarray} \label{eq:state-transition-result}
x[k+1] & = &  \overbrace{\left(\begin{array}{ccc} %
                     T & \frac{t^2}{2} & \frac{T^3}{6} \\ %
                     0 & T             & \frac{T^2}{2} \\ %
                     0 & 0             & T %
               \end{array}\right)}^{=: A} x[k]
             + \overbrace{\left(\begin{array}{ccc} %
                      \frac{T^3}{6} \\ %
                      \frac{T^2}{2} \\ %
                      T %
               \end{array}\right)}^{=: B} \cdot u_x[k]
\end{eqnarray}

\subsection{Controlling the dynamic
system}\label{controlling-the-dynamic-system}

To control this dynamic system we need to determine an adequate control
input $u_x$ to realize the reference ZMP trajectory. A performence index
$J_x$ for a given control input $u_x$ is needed to formalize what a
``good'' control input would be. A naive performence index could be:

\begin{equation}
J_x[k] := (p^{ref}_x[k] - p_x[k])^2
\end{equation}

To minimize it, we need to find $u_x$ for which $p_x = p^{ref}_x$. By
substituting $p_x[k]$ with \ref{eq:zmp-x-output} and $x[k]$ with
\ref{eq:state-transition-result} this yields:

\begin{equation}
u_x[k] = \frac{p^{ref}_x[k+1] - C \cdot A \cdot x[k]}{C \cdot B} = \frac{p^{ref}_x[k+1] - (1, T, \frac{1}{2} T^2 -\frac{z_c}{g}) \cdot x[k]}{\frac{1}{6}T^3 - \frac{z_c}{g} T}
= \frac{p^{ref}_x[k+1] - p_x[k] - T \dot{c_x}[k] - \frac{1}{2} T^2 \ddot{c_x}[k]}{\frac{1}{6}T^3 - \frac{z_c}{g} T}
\end{equation}

To analyse the behaviour of this control law for $u_x$ we simulate the
rapid change of reference ZMP when changing the support foot.
\todo{insert plot}

As you can see the reference ZMP is perfectly tracked. However, the CoM
does not behave as expected. To achive the required ZMP position the CoM
will be \emph{accelerated indefinietly} in the opposite direction.
Clearly this is not desired and will lead to falling on a real robot. A
more sophisticated performence index is needed. To eventually reach a
stable state at which the CoM comes to rest, the performence index
should include a state feedback. Also note the large jerk that is
applied to the system when the reference ZMP position changes rappidly.
In a real mechanical system large jerks will lead to oszillations, which
will disturbe the system. Thus the performence index should also try to
limit the applied jerk.

Another problem is caused by the very nature of a controller: The
controller starts to act \emph{after} we have a deviation from our
reference ZMP trajectory. Trying make this lag as small as possible can
lead to very high velocities, which might not be realizable by motors of
a robot. However we have at least limited knowledge of the future
reference trajectory. This knowledge can be leveraged by using Preview
Control, which considers the next $N$ timesteps for computing the
performence index.

\todo{add citation katayama} Kajita et. al. use a performence index
proposed by Katayama et. al. to solve all of the problems above:

\begin{equation}
J_x[k] = \sum^{\infty}_{i=k} Q_e e[i]^2 + \Delta x[i]^T Q_x \Delta x[i] + R \Delta u_x[i]^2
\end{equation}

$Q_e$ is the error gain, $Q_x$ a symmetric non-negative definite matrix
(typically just a diagonal matrix) to weight the components of
$\Delta x[i]$ differently and $R > 0$. Conveniently Katayama also
derived an optimal controller for this performence index, which is given
by:

\begin{equation}
u[k] = -G_i \sum^k_{i=0} e[k] - G_x x[k] - \sum^N_{j=1} G_p p^{ref}_x[k + j]
\end{equation}

The gains $G_i, G_x, G_p$, can be derived from the parameters of the
performence index. Since the calculation is quite elaborate we refer to
the cited article by Katayama p.~680 for more details.

\section{Implementation}\label{implementation}

\todo{block diagramm of architechture} To generate walking patterns
based on the ZMP preview control methode, the approach from Kajita was
implemented in a shared library. A front-end was developed to easily
change parameters, visualize and subsequently export the trajectory to
the MMM format. The implementation was build on a previous
implementation, which was refactored, extended and tuned with respect to
results from the dynamics simulation.

The pattern generator makes extensive usage of Simoxs VirtualRobot, for
providing a model of the robot and the associated task of computing the
forward- and inverse kinematics.

Generating a walking pattern consists of multiple steps. First the foot
positions are calculated. These are used to derive the reference ZMP
trajectory which is feed into the zmp preview controller. From that the
CoM trajectory is computed. The CoM trajectory and feet trajectories are
then used to compute the inverse kinematics. The resulting joint
trajectory is displayed in the visual front-end and can be exported.
Each step is contained in dedicated modules that can be easily replaced,
if needed. We will outline the implementation of each module seperately.

\subsection{Generating foot
trajectories}\label{generating-foot-trajectories}

To generate the foot trajectories several parameters are needed:

\todo{table with used parameters}

\begin{description}
\item[Step height $h$]
Maximum distance between the foot sole and the floor
\item[Step length $l$]
Distance in anterior direction ($y$-Axis) between the lift-off point and
the touch-down point
\item[Step width $w$]
Distance in lateral directoin ($x$-Axis) between both TCP on the feet
\item[Single support duration $t_{ss}$]
Time the weight of the robot is only support by exactly one foot
\item[Dual support duratoin $t_{ds}$]
Time the weight of the robot is supported by both feet
\end{description}

\subsubsection{Walking straight}\label{walking-straight}

Since the foot trajectories of a humanoid walking have a cyclic nature,
we only need three different foot trajectories that can be composed to
arbitrarily long trajectories: Two transient trajectories for the first
and last step respectively and a cyclic motion that can be repeated
indefinetly. We can use the same trajectories for both feet, as they are
geometrically identical. Each foot trajectory starts with swing phase
and a resting phase. The trajectory in $y$ and $z$ direction is computed
by a 5th order polynomial that assures the velocities and accelerations
are approaching zero at the lift-off and touch-down point. The first and
last step only have half of the normal step length, since the trajectory
is starting and ending from a dual support stance, where both feet are
placed parallel to eachother. Each trajectory is encoded as a
$6 \times N$ matrix, each column containing cartesian coordinates and
roll, pitch and yaw angles.

\subsubsection{Walking on a circle}\label{walking-on-a-circle}

Much of the general structure of the foot trajectory remains the same as
for walking straight. However instead of specifing the step length, it
is implicitly given by the segment of the cricle that should be
traversed and the number of steps. So extra care needs to be taken to
specify enough steps so that the generated foot positions are still.
Each foot needs to move on a circle with radius
$r_{inner} = r - \frac{w}{2}$ or $r_{outer} = r + \frac{w}{2}$ depending
which foot lies in the direction of the turn. The movement in
$z$-direction remains unaffected. However the movement in the $xy$-plane
is transformed to follow the circle for the specific foot.
\todo{Current implementation does effectively that, but is actually a hack. Needs seperate trajectories for left/right}
The same polynomial that was previously used for the $y$-direction is
now used to compute the angle on the corresponding circle and the $x$
and $y$ coordinates are calculated acordingly. The foot orientation is
computed from the tangential (y-Axis) and normal (x-Axis) of circle the
foot follows.

\subsubsection{Balancing on one foot}\label{balancing-on-one-foot}

To test push recovery from single support stance a special pattern was
needed. To generate this another footstep planer was implemented that
generates a trajectory for standing on one foot. Starting from dual
support stance, the swing leg is moved in vertical direction until the
usual step height is achieved. Additionally the foot is moved in lateral
direction to half the step width. This reduces the necessary upper body
tilt to compensate the inbalance. For the last step the inverse movement
is performed to get back into dual support stance. This methode could be
extended to walk by setting the next support foot in a straight line
before the current support foot. The swing foot would need to be moved
in an arc in lateral direction to avoid self-collisions.
\todo{It is easy to extent this: DO IT.}

\subsection{ZMP reference generation}\label{zmp-reference-generation}

As an input for the ZMP preview control, we need a reference ZMP
movement that corresponds with the foot trajectory. The reference
generator receives a list of intervals associated with the desired
support stance and foot positions as input. In single support phase, the
reference generator places the ZMP in the center of the support polygone
of the corresponding foot. Since the support polygone is convex, the
center is the point furthest away from the border of the polygone. Thus
it should guarantee a maxium of stability with regard to possible ZMP
errors. In dual support phase, the reference generator shifts the ZMP
from the previous support foot to the next support foot. Kajita et. al.
suggest using a poylnomial to interpolate the ZMP positions between the
feet. However a simple step function
$\sigma(t) = \left\{\begin{array}{lr}p_1 & t \leq t_0 \\ p_2 & t > t_0 \end{array}\right.$
seems to suffice as well. Since the the touch-down of the swing foot
might have a small lag, it is important that $t_0$ is the middle of the
dual support phase. This assures we do not start to move the ZMP too
early.

\subsection{ZMP Preview Control}\label{zmp-preview-control}

This module implements the methode described by Kajita et. al. and uses
the methode outlined by Katayama et. al. to compute the optimal control
input $u[k]$. Since it is computational feasable, the preview periode
consists of the full reference trajectory. For an online usage of this
methode, this could be reduced to a much smaller sample size.
\todo{Add timings, implement configurable preview periode} Using the
system dynamics described by \ref{eq:state-transition-result} the CoM
trajectory, velocity and acceleration can be computed. The
implementation makes heavy use of Eigen, a high performence linear
algebra framework that uses SIMD instructions to speed up calculations.
Thus thus a calculation time of \fixme{calculation time} could be
achieved.

\subsection{Inverse Kinematics}\label{inverse-kinematics}

Using the foot trajectories and CoM trajectory the actual resulting
joint angles need to be calculated. Since the kinematic model that is
used has a total of \fixme{DOF} degrees of freedom, we need to reduce
the number of joints that are used to a sensible value. For walking only
the joints of the legs and both the torso roll and pitch joints are
used. All other joints are constrained to static values that will not
cause self-collisions (e.g.~the arms are slightly extended and do not
touch the body). For comptuing the IK additional constraints where
added, to make sure the robot has a sensible pose: The chest should
always have an upright position and the pelvis should always be parallel
to the floor. To support non-straight walking, the pelvis and chest
orientation should also follow the walking direction. Thus the following
methode to compute the desired chest and pelvis orientation is used:

\begin{enumerate}
\def\labelenumi{\arabic{enumi}.}
\item
  Compute walking direction $y'$ as normed mean of y-Axis of both feet:
  $y' := \frac{y_{left} + y_{right}}{|y_{left} + y_{right}|}$
\item
  Both should have an upright position $z' := (0, 0, 1)^T$
\item
  Compute $x'$ as the normal to both vectors: $x' := y' \times z'$
\item
  Pose $R'$ is given by $R' = (x', y', z')$
\end{enumerate}

A special property of the model that was used for computing the inverse
kinematics, is that TCP of the left leg was chosen as root node. Since
we can specify the root position freely, that removes the need of
solving for the left foot pose. Thus the following goals need to be
satisfied by the inverse kinematics:

\begin{enumerate}
\def\labelenumi{\arabic{enumi}.}
\item
  Chest orientation
\item
  Pelvis orientation
\item
  CoM position
\item
  Right foot pose
\end{enumerate}

To solve the inverse kinematics a hiearchical solver was used to solve
for that goals in the given order. It was observed that specifing a good
target height for the CoM is of utmost importance for the quality of the
IK. Specifing the CoM height too height or too low can lead to the
effective loss of degrees of freedom.

\todo{Maybe a more theoretical explaination?}

\subsection{Trajectory Export}\label{trajectory-export}

The trajectory was exported in open MMM trajectory format. The format
was extended to export additional information useful for debugging and
controlling the generated trajectory. That means besides the joint
values and velocites the trajectory also includes the CoM and ZMP
trajectory that was used to derive them. Also information about the
current support phase is saved. For convinience the pose of chest,
pelvis, left and right foot are exported as homogenous matrices as well.
This was done to save the additional step of computing them again from
the exported joint trajectory for the stabilizer and also reduce an
additional error source.

\todo{Maybe doing the FK now would be better and more versatile, since we could feed normal MMM trajectories in the stabilizer}

\section{Dynamic simulation}\label{dynamic-simulation}

To evaluate the generated trajectories a simulator for the dynamics was
developed. The simulator was build on the SimDynamics framework that is
part of Simox. SimDynamics uses Bullet Physics as underlying physics
framework. A big part of the work on the simulator was spend on
configuring the parameters and finding flaws in the physics simulation.
Thus the simulator includes a extensive logging framework that measures
all important parameters of the simulation. For visulizing and analysing
the measurement the Open Source tools IPython, numpy and pandas where
used.

\subsection{Practical challenges of physics
simulation}\label{practical-challenges-of-physics-simulation}

While walking only the feet of the robot are in contact with the ground.
Thus the stability of the whole robots depends on the contact of the
feet with the floor. Especially in single support phase that area is
very small with regard to the size of the robot. For that reason the
accuracy of ground contatc forces and friction is of utmost importance
for the quality of the simulation. In general three classes of errors
need to be elimnated to get a good simulation:

\begin{enumerate}
\def\labelenumi{\arabic{enumi}.}
\item
  Incorrectly configured parameters, such as fictions coefficent and
  contact thresholds
\item
  Numerical errors
\item
  Inherent errors of the methode
\end{enumerate}

As outline in the section about discrete time dynamic simulation, the
physics of the system are formulated as input forces and constrains that
need to be solved for the constraint forces. Since bullet uses an
iterative approach that solves each constraint independently, it is of
utmost importance to use a sufficient amount of iterations for each
simulation step. Another important parameter is the timestep of each
simulation step. Through experimental evaluation a simulation with 2000
solver iterations and a timestep size of 1ms was sufficiently stable.
However since the number of iterations is very high and a lot of
timesteps are calculated during the simulation, numeric errors become
significant. That made is neccessary to enable using double precision
floating point numbers for the values used during simulation.

To decide which contact constrains are active for which points, Bullet
must solve for object collisions. Depending on the objects involved
different algorithms are used to calculate the contact points. Major
gains in accuracy could be observed by replacing the feet and the floor
with simple box shapes, instead using mesh based models.

\subsection{Simulating walking
patterns}\label{simulating-walking-patterns}

The simulator was designed to load arbitrary motions in the MMM format
and replay them. Additional stabilization algorithms can be applied
depending on additional information provided in the MMM motions.
\todo{Make sure you can really load vanilla MMM trajectories without crashing}

Even during simple playback of a trajectory, the dynamics need to be
taken into account. Simply applying the joint values at the given point
in time, will lead to large velocity impulses, which will in turn cause
large accelerations and jerks. This will cause large oscillations, which
in turn result in destabilizing disturbances. Thus interpolation between
the joint angles of two frames is needed. This was implemented using
cubic splines, that also ensure that the velocity is continous.
Disturbances due to the simulation will cause position errors in the
joints. To fix that PID based motor controllers were added to
SimDynamics that control the motor velocites to compensate position
errors.

\chapter{Controllers to stabilize a
trajectory}\label{controllers-to-stabilize-a-trajectory}

Theory Implementation Evaluation

\chapter{Push recovery}\label{push-recovery}

Theory Implementation Evaluation

\chapter{Results}\label{results}

Make sure that somewhere, either here or in the evaluation sections, you
show how you plotted the desired vs.~real zmp, even the phantom robot if
you want, and the graphics that you generated.

\chapter{Conclusions}\label{conclusions}

things to improve summary of work done and results

\bibliographystyle{dinat} %-Deutsch

\bibliography{ba}

\end{document}
